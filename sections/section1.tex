\section{Wstęp}
\label{sec:wstep}
Celem pracy było zbadanie możliwości integracji modułów \texttt{ESP8266} z koncepcją  Internetu Rzeczy (ang. Internet of Things, IoT). Cel osiągnięto poprzez stworzenie infrastruktury na którą składają się urządzenia ESP8266 ze szczególnym uwzględnieniem \texttt{topologii}, \texttt{niezawodności} jak i \texttt{uniwersalności} systemu. W pracy zbudowany został rozproszony system bazujący na układach ESP8266, realizujący komunikację z \texttt{mobilnym serwerem}.

Zaimplementowano dwa rozwiązania tego problemu, jedno bazujące na działającym, na każdym urządzeniu serwisie \texttt{telnet}, natomiast drugie na web serwisie \texttt{CoAP}. Zweryfikowana została hipoteza o możliwości integracji modułów ESP8266 z koncepcją internetu rzeczy i możliwością dynamicznego rozszerzenia stworzonego systemu.

\subsection{Motywacja - charakterystyka problemu}
Istotą Internetu rzeczy (ang. Internet of Things, IoT) jest możliwość połączenia w sieć każdego rodzaju urządzeń. Koncepcja ta zakłada stworzenie takiej infrastruktury, która połączy ze sobą urządzenia codziennego użytku, w calu dostarczenia nowych funkcjonalności i usług. W nadchodzącym czasie będziemy mieli możliwość obserwacji jak fundamentalnie zmienia się możliwość integracji świata fizycznego, ze światem urządzeń cyfrowych. 

Współcześnie istnieje bardzo wiele urządzeń komunikujących się ze sobą, tworzących swoistą sieć, nazwaną Internet rzeczy\cite{iot-art}. Jest to koncepcja, wedle której jednoznacznie identyfikowalne przedmioty mogą pośrednio lub bezpośrednio gromadzić, przetwarzać lub wymieniać dane za pośrednictwem sieci komputerowej. 

Na rynku istnieje kilka rozwiązań pozwalających na zarządzanie tymi urządzeniami. Są to między innymi opisane szerzej w \autoref{sec:existing-systems} Contiki OS, Windows 10 IoT Core, Raspberry Pi czy Zetta. Celem pracy było stworzenie systemu o podobnej funkcjonalności, pozwalającej użytkownikom zarządzać, zbierać dane, sterować różnymi urządzeniami. Użytkownicy mają mieć wygodny i prosty w obsłudze interfejs. 

Przyczyna powstania pomysłu, oraz problem i cel projektu, mając na uwadze przedstawione powyżej \texttt{tezy}, można formalnie przedstawić następująco:

\begin{itemize}
	\item \texttt{Przyczyna:} Diametralnie zwiększająca się ilość urządzeń wchodzących w skład Internetu rzeczy\cite{iot-art}.
	\item \texttt{Problem:} Stworzenie uniwersalnego interfejsu do efektywnego zarządzania tymi urządzeniami.
	\item \texttt{Rozwiązanie:} Dostarczenie modelu działającego w uproszczonym środowisku, dostarczającego wysokopoziomowy interfejs.
\end{itemize}

\subsection{Przegląd istniejących rozwiązań}
\label{sec:existing-systems}
Możliwość zdalnego konfigurowania urządzeń wchodzących w skład Internetu rzeczy przez użytkownika końcowego, bądź też samodzielnego działania układu w koncepcji IoT posiada niezliczoną ilość zastosowań w wielu obszarach, takich jak:

 \begin{itemize}
	\item Mieszkania,
	\item Miasta,
	\item Przemysł,
	\item Transport
\end{itemize}

Istnieje kilka projektów adresujących ten problem. Przykładowe z nich to Contiki OS, Windows 10 IoT Core, Raspberry Pi oraz Zetta. Zapoznanie się z nimi, z rozwiązaniami w nich stosowanymi oraz z oferowanymi przez nie interfejsami pozwoliło uniknąć wielu błędów i problemów występujących podczas projektowania własnego systemu.

\paragraph{Contiki OS}\cite{contiki-www} (The Open Source OS for the Internet of Things) \texttt{Contiki} łączy małe, tanie i energooszczędne mikrokontrolery z internetem. Contiki wspiera w pełni standard \texttt{IPv6} i \texttt{IPv4}, wraz z energooszczędnymi bezprzewodowymi standardami: \texttt{6lowpan}, \texttt{RPL}, \texttt{CoAP}. 

Contiki oferuje łatwe i szybkie tworzenie oprogramowania, aplikacje są pisane w czystym C, wraz z symulatorami \texttt{Cooja} systemy te mogą być emulowane przed wrzuceniem ich na hardware. Środowisko jest dostępne i całkowicie darmowe, co jest dużym plusem oprogramowania Contiki.

Contiki można uruchomić na wielu energooszczędnych, bezprzewodowych urządzeniach, takich jak\cite{contiki-hardware-www}: \texttt{CC2538}, \texttt{Sensortag}, \texttt{CC2650}

\paragraph{Windows 10 IoT Core}\cite{windows-iot-www} to również system operacyjny, jest to wersja Windows 10, która została zoptymalizowana dla małych urządzeń bez wyświetlacza, oraz które są uruchamiane na \texttt{Raspberry Pi} 2 i 3, \texttt{Arrow DragonBoard 410c} i \texttt{MinnowBoard MAX}. 
Windows 10 IoT Core wykorzystuje bogate, uniwersalne \texttt{API - Universal Windows Platform (UWP)}. UWP API pozwala na tworzenie aplikacji, która może być używana na wielu urządzeniach, telefonach lub komputerach i udostępnia dostęp do tysięcy urządzeń Windows, które można wykorzystywać w projekcie.

Windows 10 IoT Core wspiera łatwe w użyciu Arduino Wiring API używane w Arduino, oraz biblioteki do bezpośredniego dostępu do urządzeń. Aby tworzyć aplikacje można również używać narzędzia Visual Studio.


\paragraph{Raspberry Pi}\cite{raspberry-www} w odróżnieniu od prezentowanych tutaj rozwiązań jest to platforma komputerowa, która może posiadać zainstalowany system Linux, jak i Windows 10 IoT Core.

\paragraph{Zetta}\cite{zetta-www} (An API-First Internet of Things Platform) jest to platforma open source zbudowana na Node.js, aby zapewnić możliwość tworzenia serwerów Internetu Rzeczy, które są uruchomione na rozproszonych komputerach. Zetta łączy REST API, WebSockets i reaktywne programowanie - idealne dla łączenia wielu urządzeń w data-intensive, aplikacje czasu rzeczywistego.

\paragraph{Podsumowanie:} Pomijając aspekty techniczne, przedstawione systemy mają kilka ważnych, łatwo widocznych cech wspólnych, które powinien posiadać również system implementowany w ramach projektu:

\begin{itemize}
	\item efektywna komunikacja np. standard Wi-Fi
	\item energooszczędność urządzeń, mimo ciągłej aktywności
	\item responsywność oraz stabilność systemu
	\item niezawodność komunikacji
\end{itemize}


