\section{Zakres funkcjonalności}
\label{sec:zakres-funkcjonalnosci}
System powinien oferować funkcjonalność typową dla innych, podobnych systemów storageowych. Rozdział ten przedstawia listę najważniejszych wymagań funkcjonalnych i niefunkcjonalnych.

\subsection{Kontekst użytkowania}

\paragraph{Aktorzy:} System przewiduje tylko jeden typ aktora - \textbf{użytkownika końcowego} (ang. \textit{end-user}). Celem użytkownika jest umieszczenie w systemie swoich plików i wykonywanie na nich podstawowych operacji CRUD (\textit{Create, Read, Update, Delete}). Mimo że system jest zbiorem heterogenicznych magazynów, dla użytkownika stanowi jednolitą całość. Użytkownik traktuje system jako usługę (\textit{storage service}), dostępną w sieci pod określonym adresem IP. System oferuje użytkownikowi możliwość skorzystania z graficznego panelu w celu wygodnego dostępu do plików lub wysłania zapytania i otrzymania odpowiedzi z użyciem protokołu HTTP poprzez dostępne API.

\paragraph{Współpracujące systemy:} System jest samodzielną aplikacją. Nie wymaga współpracy z innymi systemami.

\subsection{Wymagania funkcjonalne}
Wymagania dotyczące funkcjonalności, stawiane systemom przechowującym pliki, zawsze są podobne. Użytkownik końcowy powinien mieć możliwość:
\begin{itemize}
	\item wykonywania operacji na własnych plikach:
	\begin{itemize}
		\item utworzenie pliku - wysłanie lokalnego pliku do systemu
		\item modyfikacja pliku - podmiana zawartości pliku istniejącego w systemie
		\item odczyt pliku - pobranie pliku w postaci ciągu bajtów
		\item usunięcie pliku - permamentne usunięcie pliku i metadanych z systemu
		\item wyszukanie pliku - otrzymanie odpowiedzi czy plik znajduje się w systemie
	\end{itemize}
	\item pobrania listy wszystkich swoich plików
	\item udostępnienia własnych plików w trybie do odczytu innym użytkownikom
\end{itemize}

Obecny projekt i wymagania nie przewidują roli administratora. Jedynym jego zadaniem będzie instalacja i konfiguracja systemu na zespole komunikujących się maszyn.

\subsection{Inne wymagania}
Pozostałe cechy, jakie powinien przejawiać system, zebrane na podstawie wizji i konsultacji z klientem:
\begin{itemize}
	\item system powinien być aplikacją rozproszoną, zdecentralizowaną, pozbawioną \textit{single point of failure}
	\item system powinien być możliwy do uruchomienia na klastrze obliczeniowym złożonym z kilku maszyn pracujących pod kontrolą różnych systemów
	\item powinna istnieć łatwa możliwość dodawania nowych węzłów do systemu (w celu zwięzszenia zasobów obliczeniowych)
	\item system powinien gromadzić i analizować informacje o akcjach użytkowników
	\item zachowanie użytkownika (wykonywane przez niego operacje) powinno mieć relany wpływ na priorytet jego zapytań
	\item system powinien oferować wygodny dostęp do systemu poprzez graficzny interfejs użytkownika dostępny z przeglądarki internetowej
	\item system powinien oferować programowy dostęp do plików porzez protokół HTTP, eksponując REST API
\end{itemize}

Oprócz wymagań \textbf{produktowych}, określono również wymaganie \textbf{organizacyjne}, dotyczące procesu wytwarzania:
\begin{itemize}
	\item wykorzystanym procesem będzie prototypowanie ewolucyjne
	\item produkt wraz z dokumentacją powinien być dostarczony najpóźniej do stycznia 2015
	\item dokumentacja powinna składać się z dokumentacji technicznej, dokumentacji procesowej oraz podręcznika użytkownika
\end{itemize}

Obranie takiej metodologii pozwoli szybko doprowadzić do dostarczenia początkowej, działającej wersji systemu, a klient będzie miał realny wpływ na proces rozwoju aplikacji.